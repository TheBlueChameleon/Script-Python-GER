\chapter{Grafische Darstellung von Daten -- die MatPlotLib}
\label{chp:Matplotlib}
\epigraph{
	Real life seems to have no plots.
}{Ivy Compton-Burnett}

Menschen sind unheimlich gut darin, Bilder zu interpretieren. Computer haben ihre Stärke in der Auswertung von rohen Zahlen. Um nun die Früchte unserer Datenverarbeitung mit Python zu ernten, wollen wir Daten als graphische Plots ausgeben. Ein einfach zu bedienendes Mittel hierzu ist die MatPlotLib bzw. das Untermodul PyPlot. Das Modul MatPlotLib bietet tatsächlich so viele Funktionen, dass damit ein eigenständiger Kurs gefüllt werden könnte. Hier soll Ihnen eine Basis gezeigt werden, mit der Sie die häufigsten Aufgaben lösen können, und auf der Sie im Selbststudium leicht aufbauen können.

\section{Grundlagen}
https://matplotlib.org/tutorials/introductory/pyplot.html


%https://matplotlib.org/api/pyplot_api.html


scatter, plot, bars, isosurface, subplots, post-edit, save files, hintbox on data usage, loglog, quiver